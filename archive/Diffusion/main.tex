
\documentclass{article} % For LaTeX2e
\usepackage{style,times}
\usepackage{amsmath,amsfonts,bm,amssymb}


%%%%% customized by shuo begin
\usepackage{amsthm,hyperref,url,dutchcal}
\usepackage{tikz,colortbl,xcolor}
\usetikzlibrary{arrows, positioning, calc, decorations.pathmorphing}
\DeclareMathOperator*{\argmax}{argmax}

\newtheorem{theorem}{Theorem}
\newtheorem{definition}{Definition}
\newtheorem{proposition}{Proposition}
\newtheorem{example}{Example}

\newcommand{\todo}[1]{\textcolor{red}{#1}}
%%%% customized by shuo end

\title{A Brief Introduction to Diffusion Models}
\author{Shuo Liu \\
Computer Science\\
Northeastern University\\
\texttt{shuo.liu2@northeastern.edu} \\
}

\iclrfinalcopy % Uncomment for camera-ready version, but NOT for submission.
\begin{document}

\maketitle

\begin{abstract}
This article introduces mathematical foundations of diffusion models.
\end{abstract}

\section{Diffusion Models}

The diffusion model originates from the molecular motion of thermodynamics, which degrades the data distribution by gradually adding Gaussian noise and then uses a learnable model to recover it. Compared to other generative models like Generative Adversarial Networks (GANs) \cite{GAN}, Variational Autoencoders (VAEs), and flow models, the diffusion model is learned by one network with high-dimensional latent variables in flexible architecture, avoiding unstable training, mode collapse, and surrogated loss. 

\subsection{Problem Formulation}

Given a dataset $X=\{x_1, x_2, \cdots x_n\}$, where each sample is drawn independently from an underlying data distribution $p(x)$, the goal of generative learning is to fit a model to the data distribution such that we can synthesize new data points at will by sampling from the distribution.

\subsection{Noise Conditional Score-Based Network (NCSN)}

The Score Matching with Langevin Dynamics (SMLD) is one of the first representative works of the diffusion model \cite{NCSN}, which utilizes iterative \textit{Langevin dynamics} to draw the samples. Langevin dynamics provides a Monte Carlo Markov Chain (MCMC) procedure to sample from a distribution $p(x)$ with only its score function $\nabla_x \log p(x)$ \cite{langevin}. It initializes the chain from an arbitrary prior distribution $x_0\sim \pi(x)$ and iterates as follows,
\begin{equation} \label{equ:langevin}
    x_{i+1} \leftarrow x_i + \epsilon \nabla_x \log p(x) + \sqrt{2 \epsilon} z_i,
\end{equation}
where $z_i \sim \mathcal{N}(0, I)$. When $\epsilon \rightarrow 0$ and $K \rightarrow \infty$, $x_K$ obtained from Equ. \ref{equ:langevin} converges to a sample from $p(x)$ under some regularity conditions.

In the perturbing process, SMLD adds a sequence of multi-scale random Gaussian noises to the original data distribution $p(x)$,
\begin{equation}
    p_{\sigma_k}(x)=\int p(y) \mathcal{N}(x; y, \sigma_k^2 I) dy, k=1, 2, \cdots, K,
\end{equation}

The SMLD then approximates the score function $\nabla_x p_{\sigma_k}(x)$ of each noise-perturbed distribution by training a Noise Conditional Score-Based Network (NCSN) $s_\theta(x, k)$. The score-matching objective of an NCSN is to minimize the weighted sum of Fisher divergences for all noise scales,
\begin{equation} \label{equ:score_matching}
    L(\theta)=\sum_{k=1}^K \lambda(k) \mathbb{E}_{p_{\sigma_k}(x)}[||\nabla_x \log p_{\sigma_k}(x) - s_\theta(x, k)||_2^2],
\end{equation}
where $\lambda(i)\in \mathbb{R}_{>0}$ is a positive weighting function ($\lambda(i)=\sigma_k^2$). After obtaining the score-based model $s_\theta(x, k)$, we can produce the samples from it by running the Langevin dynamics for $k=K, K-1, \cdots, 1$ in sequence (so-called annealed Langevin dynamics).

\subsection{Denoising Diffusion Probabilistic Model (DDPM)}

Compared to NCSN which uses the score-based function (though convertible), the Denoising Diffusion Probabilistic Model (DDPM) reconstructs the samples from the noise according to \textit{variational inference}. 

In the forward chain, the DDPM gradually perturbs the raw data distribution $x_0 \sim q(x_0)$ to converge to the standard Gaussian distribution $q(x_k)$,
\begin{equation}
    \begin{gathered}
    q(x_k|x_{k-1})=\mathcal{N}(x_k; \sqrt{1-\beta_k} x_{k-1}, \beta_k I), \\ 
    q(x_{1:K}|x_0)=\prod_{t=1}^T q(x_k|x_{k-1}),
\end{gathered}\label{equ:diffusion_forward}
\end{equation}
where $\beta_k \in (0, 1)$ is the coefficient of noise added at step $t$. On the other hand, the reverse chain seeks to train a parameterized Gaussian transition kernel with $\theta$ to recover the data distribution,
\begin{equation}
    \begin{gathered}
    p_\theta(x_{k-1}|x_k)=\mathcal{N}(x_{k-1}; \mu_\theta(x_k, k), \sigma_\theta(x_k, k)I), \\ 
    p_\theta(x_{1:K}|x_k)=\prod_{t=1}^T p_\theta(x_{k-1}|x_k).
\end{gathered}\label{equ:diffusion_reverse}
\end{equation}
Our objective is to estimate the maximum likelihood of original distribution $p_\theta(x_{0:K})$ by maximizing the variational lower bound $\mathbb{E}_q[\frac{p_\theta(x_{0:K})}{q(x_{1:K|x_0})}]$. After parameterization \cite{hodiffusion}, we optimize the loss function,
\begin{equation}
    \mathcal{L(\theta)} = \mathbb{E}_{x_0 \sim q(x_0), \epsilon \sim \mathcal{N}(0, I), k} \vert \vert\epsilon - \epsilon_
    \theta(x_k, k) \vert \vert ^2_2,
\end{equation}
where $\epsilon_\theta$ estimates the noise input. Once trained, we can sample $x_0$ from Equ. \ref{equ:diffusion_reverse}.

\subsection{Score-based Generative Model (SGM)}

The Score-based Generative Model (SGM) using \textit{Stochastic Differential Equation} (SDE) \cite{song2021scorebased} describes the diffusion process in continuous time steps with a standard Wiener process, which unifies NCSN and DDPM. The forward diffusion process in infinitesimal time can be formally represented as
\begin{equation}
    dx = f(x, k) dk + \sigma(k) dw,
\end{equation}
where $w$ denotes a standard Winener process and $\sigma(\cdot)$ denotes the diffusion coefficient, which is supposed to be a scalar independent of $x$. The reserve SDE describes the diffusion process running backward in time to generate new samples from the known prior $x_k$ \cite{reversesde}, which is
\begin{equation}
    dx = [f(x, k) - \sigma(k)^2 \nabla_x \log p_k(x)] dk + \sigma(k) d\bar{w},
\end{equation}
which incorporate a backward induction score function $\log p_k(x)$. Similar to NCSN, the score function can be approximated using a step-dependent score-based model $s_\theta(x, k)$ with the score-matching optimization objective Equ. \ref{equ:score_matching}.


\clearpage
\bibliography{ref}
\bibliographystyle{ref}
\newpage
\appendix
\section{Omitted Proofs}

This section shows the omitted (but non-trivial) proofs.

\subsection{Proof Theorem~\ref{them:PG}}
\begin{proof}
The gradient of $V$ function can be written in terms of $Q$ function,

{\footnotesize
    \begin{equation}
        \begin{aligned}
            \nabla V^{\pi}(s) &= \nabla\left [ \sum_a \pi(a|s) Q^\pi (s,a)\right ] \\
            &=\sum_a\left [ \nabla\pi(a|s) Q^\pi (s,a) +\pi(a|s) \nabla Q^\pi (s,a) \right] \\
            &=\sum_a\left [ \nabla\pi(a|s) Q^\pi (s,a) +\pi(a|s) \nabla \sum_{s'}P(s'|s, a)(r+V^\pi(s')) \right] \\
            &\stackrel{\text{(i)}}{=}\sum_a\left [ \nabla\pi(a|s) Q^\pi (s,a) +\pi(a|s) \sum_{s'}P(s'|s, a)\nabla V^\pi(s') \right ], \\
        \end{aligned}\nonumber
    \end{equation}
}
where we can have the derivation (i) since $r$ only depends on the environment dynamics.

Let $\phi(s) = \sum_a \nabla\pi(a|s) Q^\pi (s,a)$. We use $\rho^\pi(s\rightarrow x, k)$ to represent the probability of transitioning from state $s$ to $x$ with policy $\pi$ after $k$ steps, i.e., $\rho^\pi(s \rightarrow s', 1)=\sum_a \pi(a|s) P(s'|s, a)$. Thus, we can unroll the recursive form as below,

{\footnotesize
    \begin{equation} 
        \begin{aligned}
            \nabla V^{\pi}(s) &=\phi(s) +\sum_a \pi(a|s) \sum_{s'}P(s'|s, a)\nabla V^\pi(s') \\
            &= \phi(s) + \sum_{s'}\rho^\pi(s \rightarrow s', 1) \nabla V^\pi(s') \\
            &= \phi(s) + \sum_{s'}\rho^\pi(s \rightarrow s', 1)\left[\phi(s') + \sum_{s''}\rho^\pi(s' \rightarrow s'', 1) \nabla V^\pi(s'')\right] \\
            &= \phi(s) + \sum_{s'}\rho^\pi(s \rightarrow s', 1) \phi(s') + \sum_{s''}\rho^\pi(s \rightarrow s'', 2) \nabla V^\pi(s'')  \\
            &= \phi(s) + \sum_{s'}\rho^\pi(s \rightarrow s', 1) \phi(s') + 
             \sum_{s''}\rho^\pi(s \rightarrow s'', 2) \phi(s'') +
            \sum_{s'''}\rho^\pi(s \rightarrow s''', 3) \nabla V^\pi(s''')  \\
            & \qquad \qquad \vdots \\
            &= \sum_{k=0}^{\infty}\sum_{x}\rho^\pi(s\rightarrow x, k)\phi(x)\\
        \end{aligned}\nonumber
    \end{equation}
}

We use $\eta(s)$ to represent the expected number of visits for $s$ in a single episode (in episodic case $\sum_s \eta(s)$ is the averaged length of an episode; in continuous case $\sum_s \eta(s)=1$). By plugging it into the object function $J$,
\begin{equation}
    \begin{aligned}
        \nabla J(\theta) &= \nabla V^\pi (s_0)   \\
        &= \sum_s \left( \sum_{k=0}^\infty \rho^\pi(s_0 \rightarrow s, k) \right) \sum_a \nabla\pi(a|s) Q^\pi (s,a) & \\
        &= \sum_s \eta(s) \sum_a \nabla\pi(a|s) Q^\pi (s,a)  \\
        &\stackrel{\text{norm}}{=} \left(\sum_s \eta(s) \right) \left(\sum_s \frac{\eta(s)}{\sum_s \eta(s)} \right )\sum_a \nabla\pi(a|s) Q^\pi (s,a) \\
        &\propto \sum_s d^\pi(s)\sum_a \nabla\pi(a|s) Q^\pi (s,a)
    \end{aligned}\nonumber
\end{equation}
\end{proof}

\subsection{Proof of Theorem~\ref{them:PG-baseline}}

\begin{proof}
We first prove PG with baseline is unbiased,
    \begin{equation}
        \begin{aligned}
            &\quad\mathbb{E}_{ d^\pi}\left [\sum_a (Q^\pi(s,a) -b(s)) \nabla\ln\pi(a|s)\right]\\
            &=\mathbb{E}_{ d^\pi}\left [\sum_a Q^\pi(s,a)\nabla\ln\pi(a|s)\right] -\mathbb{E}_{ d^\pi}\left [\sum_a b(s) \nabla\ln\pi(a|s)\right]\\
            &= \mathbb{E}_{ d^\pi}\left [\sum_a Q^\pi(s,a)\nabla\ln\pi(a|s)\right] - \mathbb{E}_{ d^\pi}\left [b(s) \nabla\sum_a \ln\pi(a|s)\right]\\
            &=\mathbb{E}_{ d^\pi}\left [\sum_a Q^\pi(s,a)\nabla\ln\pi(a|s)\right] - \mathbb{E}_{ d^\pi}\left [b(s) \nabla 1\right]\\
            &=\mathbb{E}_{ d^\pi}\left [\sum_a Q^\pi(s,a)\nabla\ln\pi(a|s)\right],
        \end{aligned} \nonumber
    \end{equation}
and the variance of PG with baseline is,
    \begin{equation}
        \begin{aligned}
            &\quad \mathbb{V}_{ d^\pi}
            \left [\sum_a (Q^\pi(s,a) -b(s)) \nabla\ln\pi(a|s)\right]\\ 
            &\stackrel{\text{(i)}}{\gtrapprox} \sum_a \mathbb{E}_{ d^\pi} \left [ \left((Q^\pi(s,a) -b(s)) \nabla\ln\pi(a|s)\right)^2\right] - \left(\mathbb{E}_{ d^\pi} \left [\sum_a (Q^\pi(s,a) -b(s)) \nabla\ln\pi(a|s)\right]\right)^2 \\
             &\stackrel{\text{(ii)}}{\approx} \sum_a \mathbb{E}_{ d^\pi} \left [\left(Q^\pi(s,a) -b(s)\right)^2\right] \mathbb{E}_{ d^\pi}\left[\left(\nabla\ln\pi(a|s)\right)^2\right] - \left(\mathbb{E}_{d^\pi} \left [\sum_a Q^\pi(s,a) \nabla\ln\pi(a|s)\right]\right)^2 \\
             &< \sum_a \mathbb{E}_{ d^\pi} \left [\left(Q^\pi(s,a)\nabla\ln\pi(a|s)\right)^2\right] - \left(\mathbb{E}_{ d^\pi} \left[\sum_a Q^\pi(s,a) \nabla\ln\pi(a|s)\right]\right)^2 \\
             &\stackrel{\text{(iii)}}{\lessapprox} \mathbb{E}_{ d^\pi} \left [\left(\sum_a Q^\pi(s,a)\nabla\ln\pi(a|s)\right)^2\right] - \left(\mathbb{E}_{ d^\pi} \left[\sum_a Q^\pi(s,a) \nabla\ln\pi(a|s)\right]\right)^2 \\
             &= \mathbb{V}_{ d^\pi} \left[ \sum_a Q^\pi(s,a)\nabla \ln \pi(a|s)\right].
        \end{aligned} \nonumber
    \end{equation}
    
    In approximations (i) and (iii), we only keep the quadratic term and omit the products, but this won't affect the property of the inequality because the deduction loss caused by $\prod_a (Q^\pi(s,a) -b(s)) \nabla\ln\pi(a|s)$ is less than the increase we compensate for $\prod_a Q^\pi(s,a) \nabla\ln\pi(a|s)$. In approximation (ii), we assume independence among the values involved in the expectation for factorization. This reveals that REINFORCE has less variance when using a baseline; and when $b(s)\approx V^\pi(s)$, the variance reaches optimal.
\end{proof}

\newpage
\section{Other Forms of PG}

Since the major purpose of this article is to introduce PPO methods from PG, we omit some other important forms of PG in the main body. Here are the complements.

\subsection{Deterministic PG}

Sometimes we hope the policy function to be deterministic to reduce the gradient estimation variance and improve the exploration efficiency for continuous action space \footnote{The deterministic PG is a special case of the stochastic PG, with $\sigma=0$ in the re-parameterization $\pi_{\mu_{\theta}, \sigma}$.} (i.e., a decision $a=\mu_{\theta}(s)$). PG for a deterministic policy in continuous action space is,
\begin{equation}\label{equ:pgthem-deterministic}
        \begin{aligned}  
            \nabla_{\theta} J(\theta) &=\nabla_{\theta} \left(\int_s d^\mu(s) V^\mu(s) ds\right)\\
            &=\nabla_{\theta} \left(\int_s d^\mu(s) Q^\mu(s,a)|_{a=\mu_{\theta}(s)}ds\right)\\
            &\stackrel{\text{(i)}}{=}\int_s d^\mu(s) \nabla_{\theta} \mu_{\theta}(s) \nabla_a Q^\mu(s,a)|_{a=\mu_{\theta}(s)} ds 
            \\
            &=\mathbb{E}_{d^\mu} [ \nabla_{\theta} \mu_{\theta}(s) \nabla_a Q^\mu(s,a)|_{a=\mu_{\theta}(s)}],
        \end{aligned}
\end{equation}
The derivation (i) the state distribution is non-differentiable w.r.t. $\theta$ (i.e., derivation (i))\footnote{A small change in $\theta$ can cause a substantial change in the trajectory, and the state visitation distribution can exhibit non-smooth behavior as a function of $\theta$.}, To guarantee enough exploration of determinant PG, We can either add noise into the policy 
\begin{equation}
    \mu'(s) = \mu_{\theta}(s) + \mathcal{N},
\end{equation} 
or learn it off-policy-ly by following a different stochastic behavior $\beta(a|s)$ policy to collect samples,
\begin{equation}\label{equ:pgthem-deterministic-offpolicy}
        \begin{aligned}  
            \nabla_{\theta} J(\theta) &=\nabla_{\theta} \left(\int_s d^\beta(s) Q^\mu(s,a)|_{a=\mu_{\theta}(s)}ds\right)\\
            &=\mathbb{E}_{d^\beta} [ \nabla_{\theta} \mu_{\theta}(s) \nabla_a Q^\mu(s,a)|_{a=\mu_{\theta}(s)}],
        \end{aligned}
\end{equation}

\subsection{Distributed PG}

Due to the efficiency of the GPU-cluster in training, some workers (machines or processes) are employed in a distributed manner to generate rollouts and compute policy gradients in PG methods \cite{brenner2023ppo}. The distributed advancement can also be extended to any PG extension, like Actor-Critic (AC), PPO, and deterministic PG methods.

\paragraph{Centralized v.s. Decentralized} These workers can either share a central parameter server or update their own weights in a decentralized manner, where aggregation techniques such as AllReduce may be utilized. Rather than merely collecting rollouts and calculating the gradient according to its replay buffer, the workers can be further decentralized into \textit{agents} with their parameters, which is closely related to PG in multi-agent setting.

\paragraph{Synchronous v.s. Asynchronous} In the centralized paradigm, weight updates can be conducted synchronously, where gradients from all workers are aggregated (typically through summation or averaging) before updating the model parameters. This ensures a globally consistent update but may introduce inefficiencies due to synchronization delays. Alternatively, asynchronous updating allows each worker to update the global parameters independently, without waiting for all gradients to be collected. This method can improve computational throughput but may lead to stale gradients and slower convergence. The difference between these 2 approaches is exemplified in Advantage Actor-Critic (A2C) and Asynchronous Advantage Actor-Critic (A3C).




\end{document}
