
\documentclass{article} % For LaTeX2e
\usepackage{style,times}
\usepackage{amsmath,amsfonts,bm,amssymb}


%%%%% customized by shuo begin
\usepackage{amsthm,hyperref,url,dutchcal}
\usepackage{tikz,colortbl,xcolor}
\usetikzlibrary{arrows, positioning, calc, decorations.pathmorphing}
\DeclareMathOperator*{\argmax}{argmax}

\newtheorem{theorem}{Theorem}
\newtheorem{definition}{Definition}
\newtheorem{proposition}{Proposition}
\newtheorem{example}{Example}

\newcommand{\todo}[1]{\textcolor{red}{#1}}
%%%% customized by shuo end

\title{A Brief Introduction to Diffusion Models}
\author{Shuo Liu \\
Computer Science\\
Northeastern University\\
\texttt{shuo.liu2@northeastern.edu} \\
}

\iclrfinalcopy % Uncomment for camera-ready version, but NOT for submission.
\begin{document}

\maketitle

\begin{abstract}
This article introduces mathematical foundations of diffusion models.
\end{abstract}

\section{Diffusion Models}

The diffusion model originates from the molecular motion of thermodynamics, which degrades the data distribution by gradually adding Gaussian noise and then uses a learnable model to recover it. Compared to other generative models like Generative Adversarial Networks (GANs) \cite{GAN}, Variational Autoencoders (VAEs), and flow models, the diffusion model is learned by one network with high-dimensional latent variables in flexible architecture, avoiding unstable training, mode collapse, and surrogated loss. 

\subsection{Problem Formulation}

Given a dataset $X=\{x_1, x_2, \cdots x_n\}$, where each sample is drawn independently from an underlying data distribution $p(x)$, the goal of generative learning is to fit a model to the data distribution such that we can synthesize new data points at will by sampling from the distribution.

\subsection{Noise Conditional Score-Based Network (NCSN)}

The Score Matching with Langevin Dynamics (SMLD) is one of the first representative works of the diffusion model \cite{NCSN}, which utilizes iterative \textit{Langevin dynamics} to draw the samples. Langevin dynamics provides a Monte Carlo Markov Chain (MCMC) procedure to sample from a distribution $p(x)$ with only its score function $\nabla_x \log p(x)$ \cite{langevin}. It initializes the chain from an arbitrary prior distribution $x_0\sim \pi(x)$ and iterates as follows,
\begin{equation} \label{equ:langevin}
    x_{i+1} \leftarrow x_i + \epsilon \nabla_x \log p(x) + \sqrt{2 \epsilon} z_i,
\end{equation}
where $z_i \sim \mathcal{N}(0, I)$. When $\epsilon \rightarrow 0$ and $K \rightarrow \infty$, $x_K$ obtained from Equ. \ref{equ:langevin} converges to a sample from $p(x)$ under some regularity conditions.

In the perturbing process, SMLD adds a sequence of multi-scale random Gaussian noises to the original data distribution $p(x)$,
\begin{equation}
    p_{\sigma_k}(x)=\int p(y) \mathcal{N}(x; y, \sigma_k^2 I) dy, k=1, 2, \cdots, K,
\end{equation}

The SMLD then approximates the score function $\nabla_x p_{\sigma_k}(x)$ of each noise-perturbed distribution by training a Noise Conditional Score-Based Network (NCSN) $s_\theta(x, k)$. The score-matching objective of an NCSN is to minimize the weighted sum of Fisher divergences for all noise scales,
\begin{equation} \label{equ:score_matching}
    L(\theta)=\sum_{k=1}^K \lambda(k) \mathbb{E}_{p_{\sigma_k}(x)}[||\nabla_x \log p_{\sigma_k}(x) - s_\theta(x, k)||_2^2],
\end{equation}
where $\lambda(i)\in \mathbb{R}_{>0}$ is a positive weighting function ($\lambda(i)=\sigma_k^2$). After obtaining the score-based model $s_\theta(x, k)$, we can produce the samples from it by running the Langevin dynamics for $k=K, K-1, \cdots, 1$ in sequence (so-called annealed Langevin dynamics).

\subsection{Denoising Diffusion Probabilistic Model (DDPM)}

Compared to NCSN which uses the score-based function (though convertible), the Denoising Diffusion Probabilistic Model (DDPM) reconstructs the samples from the noise according to \textit{variational inference}. 

In the forward chain, the DDPM gradually perturbs the raw data distribution $x_0 \sim q(x_0)$ to converge to the standard Gaussian distribution $q(x_k)$,
\begin{equation}
    \begin{gathered}
    q(x_k|x_{k-1})=\mathcal{N}(x_k; \sqrt{1-\beta_k} x_{k-1}, \beta_k I), \\ 
    q(x_{1:K}|x_0)=\prod_{t=1}^T q(x_k|x_{k-1}),
\end{gathered}\label{equ:diffusion_forward}
\end{equation}
where $\beta_k \in (0, 1)$ is the coefficient of noise added at step $t$. On the other hand, the reverse chain seeks to train a parameterized Gaussian transition kernel with $\theta$ to recover the data distribution,
\begin{equation}
    \begin{gathered}
    p_\theta(x_{k-1}|x_k)=\mathcal{N}(x_{k-1}; \mu_\theta(x_k, k), \sigma_\theta(x_k, k)I), \\ 
    p_\theta(x_{1:K}|x_k)=\prod_{t=1}^T p_\theta(x_{k-1}|x_k).
\end{gathered}\label{equ:diffusion_reverse}
\end{equation}
Our objective is to estimate the maximum likelihood of original distribution $p_\theta(x_{0:K})$ by maximizing the variational lower bound $\mathbb{E}_q[\frac{p_\theta(x_{0:K})}{q(x_{1:K|x_0})}]$. After parameterization \cite{hodiffusion}, we optimize the loss function,
\begin{equation}
    \mathcal{L(\theta)} = \mathbb{E}_{x_0 \sim q(x_0), \epsilon \sim \mathcal{N}(0, I), k} \vert \vert\epsilon - \epsilon_
    \theta(x_k, k) \vert \vert ^2_2,
\end{equation}
where $\epsilon_\theta$ estimates the noise input. Once trained, we can sample $x_0$ from Equ. \ref{equ:diffusion_reverse}.

\subsection{Score-based Generative Model (SGM)}

The Score-based Generative Model (SGM) using \textit{Stochastic Differential Equation} (SDE) \cite{song2021scorebased} describes the diffusion process in continuous time steps with a standard Wiener process, which unifies NCSN and DDPM. The forward diffusion process in infinitesimal time can be formally represented as
\begin{equation}
    dx = f(x, k) dk + \sigma(k) dw,
\end{equation}
where $w$ denotes a standard Winener process and $\sigma(\cdot)$ denotes the diffusion coefficient, which is supposed to be a scalar independent of $x$. The reserve SDE describes the diffusion process running backward in time to generate new samples from the known prior $x_k$ \cite{reversesde}, which is
\begin{equation}
    dx = [f(x, k) - \sigma(k)^2 \nabla_x \log p_k(x)] dk + \sigma(k) d\bar{w},
\end{equation}
which incorporate a backward induction score function $\log p_k(x)$. Similar to NCSN, the score function can be approximated using a step-dependent score-based model $s_\theta(x, k)$ with the score-matching optimization objective Equ. \ref{equ:score_matching}.


\clearpage
\bibliography{ref}
\bibliographystyle{ref}
\newpage
\appendix

\section{Deep Q-Network}

Deep Q-Network (DQN) approximates the Q-function with a target network to stabilize training.

\begin{algorithm}[H]
    \centering 
    \caption{Deep Q-learning with Experience Replay}\label{alg:dqn}
    \begin{algorithmic}[1]
    \State \textbf{Initialize}: parameter $\theta$ for Q-network $Q_{\theta}$, target Q-network $\theta'\leftarrow \theta$, replay memory $\mathcal{D}$
    \For{each episode}
        \State Choose $s_0$ from $\mathcal{S}$ as initial state
        \For{$t=0$ to $T-1$}
        \State Take action $a_t$ according to policy derived by $Q$ (e.g. $\epsilon$-greedy)
        \State Observe reward $r_{t+1}$ and next state $s_{t+1}$
        \State Store transition $(s_t, a_t, r_{t+1}, s_{t+1})$ into $\mathcal{D}$
        \For{each mini-batch of $N$ transitions $\{s_i, a_i, r_{i+1}, s_{i+1}\}$}
        \State $\delta_i\leftarrow \underbrace{r_{i+1} +\gamma \max_{a'} Q_{\theta'}(s_{i+1}, a')}_{\text{target Q prediction}} - Q_{\theta}(s_i, a_i)$
        \State $\theta \leftarrow \theta + \frac{1}{N} \sum_i \alpha \delta_i \nabla_{\theta} Q_{\theta}(s_i, a_i)$
        \EndFor
        \State $\theta'\leftarrow \tau \theta' + (1-\tau) \theta$
        \EndFor
    \EndFor
\State \Return value function $Q$
\end{algorithmic}
\end{algorithm}

%\section{Theorems}
%
%\subsection{Banach Fixed-Point Theorem}
%
%\begin{theorem} \label{them:banach}
%Let $(X, d)$ be a non-empty complete metric space with a contraction mapping $T : X \rightarrow X$. Then $T$ admits a unique \textit{fixed-point} $x^* \in X$ (i.e. $T(x^*) = x^*$). Furthermore, $x^*$ can be found as follows: start with an arbitrary element $x_0 \in X$ and define a sequence $(x_n)_{n \in \mathbb{N}}$ by $x_n = T(x_{n-1})$ for $n \geq 1$. Then $\lim_{n \to \infty} x_n = x^*$.
%\end{theorem}
%
%The proof can find in \cite{szepesvari2010algorithms}
%
%\subsection{Stochastic Approximation Theorem}
%
%\section{Examples}
%
%\subsection{Baird's Counterexample}
%
%The star problem.


\end{document}
