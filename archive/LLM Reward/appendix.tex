\newpage
\appendix
\section{LLM Terminology}

\begin{table}[h!]
    \renewcommand{\arraystretch}{1.5} % Increases line height
    \small % Reduces font size
    \centering
    \label{tab:terminology}
    \begin{tabular}{lp{10cm}} % Adjust column width if needed
        \hline \hline
        \textbf{Term} & \textbf{Definition} \\
        \hline
        \texttt{Agent} & An autonomous entity capable of perceiving the environment, taking actions to influence it, and proactively pursuing goals based on internal decision-making. \\
        \texttt{LLM Alignment} & Ensuring that large language models behave in accordance with human intentions, ethical guidelines, and societal values. \\
        \texttt{Prompt Engineering} & Designing and refining input prompts to optimize the performance of a large language model, ensuring it generates accurate, relevant, and useful responses. \\
        \texttt{Reward Modeling} & The process of defining and optimizing reward signals to train LLMs to exhibit desired behaviors. \\
        \texttt{Hallucination} & When an LLM generates false or misleading information that is not supported by its training data. \\
        \texttt{Jailbreaking} & Manipulating prompts to bypass safety mechanisms and make an LLM generate restricted content. \\
        \texttt{Chain-of-Thought Prompting} & A prompting technique that encourages the model to break down reasoning into intermediate steps, improving complex problem-solving accuracy. \\
        \hline \hline
    \end{tabular}
    \caption{Terminology of LLM research.} % Optional caption
\end{table}

